% Gemini theme
% https://github.com/anishathalye/gemini

\documentclass[final]{beamer}

% ====================
% Packages
% ====================

\usepackage[T1]{fontenc}
\usepackage{lmodern}
% Lengths are in centimeters
\usepackage[size=a1, orientation=portrait, scale=1.0]{beamerposter}
\usetheme{gemini}
\usecolortheme{wtbarnes}
\usepackage{graphicx}
\usepackage{booktabs}
\usepackage{tikz}
\usepackage{pgfplots}
\pgfplotsset{compat=newest}
\usepackage{anyfontsize}
\usepackage{multicol}
\usepackage[numbers]{natbib}
\usepackage{import}
\usepackage{siunitx}
\usepackage{float}
\usepackage{amsmath}

% ====================
% Colors
% ====================
\definecolor{C0}{HTML}{1f77b4}
\definecolor{C1}{HTML}{ff7f0e}
\definecolor{C2}{HTML}{2ca02c}
\definecolor{C3}{HTML}{d62728}
\definecolor{C4}{HTML}{9467bd}
\definecolor{C5}{HTML}{8c564b}

% ====================
% Custom Commands
% ====================
% This is to work around a bug in matplotlib that leaves a command undefined when
% using pgf figures: https://github.com/matplotlib/matplotlib/issues/27907
\def\mathdefault#1{\displaystyle #1}
\newcommand{\emd}{\textup{EM}(T)}
\DeclareMathOperator*{\argmax}{argmax}

% ====================
% Lengths
% ====================

% If you have N columns, choose \sepwidth and \colwidth such that
% (N+1)*\sepwidth + N*\colwidth = \paperwidth
\newlength{\sepwidth}
\newlength{\colwidth}
\setlength{\sepwidth}{0.013\paperwidth}
\setlength{\colwidth}{0.48\paperwidth}
\newcommand{\separatorcolumn}{\begin{column}{\sepwidth}\end{column}}

% ====================
% Title
% ====================
\title{Diagnostics of Coronal Heating in Active Regions over Multiple Solar Rotations}
\author{
  W. T. Barnes \inst{1}\textsuperscript{,}\inst{2} \and
  S. J. Bradshaw \inst{3} \and
  E. M. Mason \inst{4} \and
  N. M. Viall \inst{2}
}
\institute[]{
  \inst{1} Department of Physics, American University \samelineand
  \inst{2} Heliophysics Science Division, NASA Goddard Space Flight Center \and
  \inst{3} Department of Physics and Astronomy, Rice University
  \inst{4} Predictive Science, Inc.
}

% ====================
% Footer (optional)
% ====================
\footercontent{
  \href{https://github.com/wtbarnes/hinode-2025-poster}{github.com/wtbarnes/hinode-2025-poster} \hfill
  Hinode 18/IRIS 16 Meeting --- London, UK --- 23--27 June 2025 \hfill
  \href{mailto:wbarnes@american.edu}{wbarnes@american.edu}
}

% ====================
% Logo (optional)
% ====================
\logoright{\includegraphics[height=7cm]{static/AU_stacked_logo.png}}
\logoleft{\includegraphics[height=8cm]{static/sunpy_logo_portrait_powered.png}}

% ====================
% Body
% ====================

\begin{document}

\begin{frame}[t]
\begin{columns}[t]
\separatorcolumn

\begin{column}{\colwidth}

  \begin{block}{Introduction}

    \begin{itemize}
      \item Heating frequency: \alert{frequency at which elemental strands in the corona are reenergized}
      \begin{itemize}
        \item High frequency (HF): waiting time between successive events less than a loop cooling time
        \item Low frequency (LF): waiting time between successive events greater than a loop cooling time
      \end{itemize}
      \item \alert{Heating frequency is variable over an active region (AR)} \citep[e.g.][]{del_zanna_evolution_2015,barnes_understanding_2021}
      \begin{itemize}
        \item Warm, periphery loops exhibit low-frequency heating \citep[e.g.][]{warren_evolving_2003}
        \item Hot, core loops more consistent with high-frequency heating \citep[e.g.][]{warren_evidence_2010}
      \end{itemize}
      \item \alert{AR properties evolve with age:} ``very hot'' emission \citep{ugarte-urra_is_2012,ugarte-urra_determining_2014}, abundance \citep{testa_coronal_2023}, $T$ and $n$ \citep{ko_correlation_2016}
      \item \alert{\textbf{Goal:}} Understand how the distribution of heating frequencies in an AR evolves with age
    \end{itemize}

  \end{block}

  \begin{block}{Identifying Active Regions Observed over Multiple Rotations}

    \begin{figure}
      \begin{columns}[T]
        \begin{column}{0.45\colwidth}
            \begin{itemize}
              \item \citet{mason_active_2023} identified NOAA classifications of many ARs which persisted over multiple solar rotations.
              \item AR coordinates identified using the Heliophysics Event Knowledgebase (HEK).
              \item \alert{Exclude ARs with C-class flares and above} $\pm\SI{6}{\hour}$ of when AR appeared at disk center.
              \item Selected ARs with corresponding ``HPW021'' EIS raster $\pm\SI{4.5}{\day}$ of when AR was at disk center.
              \item Require $\ge\SI{60}{\percent}$ overlap between $(\ang{;6;},\ang{;6;})$ AR bounding box and de-rotated EIS raster.
              \item Identified three ARs (\alert{12303, 12488, 12682}) observed on \textcolor{C0}{first} and \textcolor{C1}{second} rotation and a fourth AR (\alert{12674}) observed after a \textcolor{C2}{third} rotation. 
            \end{itemize}
          \caption{Positions of ARs for all rotations. NOAA number color indicates rotation, with each \textcolor{C0}{first} rotation denoting a distinct AR (read L to R). Solid boxes indicate $(\ang{;6;},\ang{;6;})$ bounding box centered on HEK AR position. Dashed boxes indicate nominal EIS raster position. Dotted boxes indicate EIS raster position de-rotated to when AR was at disk center.}
          \label{fig:aia_eis_fov}
        \end{column}
        \begin{column}{0.55\colwidth}
          \import{figures/}{aia_eis_fulldisk_bbox.pgf}
        \end{column}
      \end{columns}
    \end{figure}
  \end{block}

  \vspace{-35pt}

  \begin{block}{AIA and EIS Observations}
    
    \begin{figure}
      \begin{columns}[T]
        \begin{column}{0.55\colwidth}
          \import{figures/}{aia_cutouts_eis_rasters.pgf}
        \end{column}
        \begin{column}{0.45\colwidth}
          \caption{AIA \SI{193}{\angstrom} and EIS Fe XII \SI{195.119}{\angstrom} observations for all ARs. Each row indicates a distinct AR and each column a subsequent rotation. Black boxes overlaid on the EIS observations indicate the regions over which the intensity is averaged when computing the $\emd$ in the panels below. AIA observations $\pm\SI{6}{\hour}$ of when the AR appeared at disk center are co-aligned to the time closest to when the AR appeared at disk center, normalized by exposure time, and corrected for degradation for the \SIlist{94;131;171;193;211;335}{\angstrom} channels. For each EIS observation, we fit each line in Table 2 of \citet{warren_systematic_2012}, except Ca XVI \SI{208.604}{\angstrom}, Ca XVII \SI{192.858}{\angstrom}, and Ar XIV \SI{194.396}{\angstrom}, using the \texttt{eispac} Python package \citep{weberg_eispac_2023}. The EIS pointing is corrected by cross-correlating the AIA \SI{193}{\angstrom} and EIS \SI{195.119}{\angstrom} images. Additionally, the radiometric calibration of \citet{del_zanna_hinode_2025} is used.} 
          \label{fig:aia_eis_intensities}
        \end{column}
      \end{columns}
    \end{figure}

  \end{block}

  \vspace{-35pt}

  \begin{block}{Deriving Emission Measure Distributions with EIS}

    \begin{figure}
      \centering
      \import{figures/}{em_distributions_1d.pgf}
      \label{fig:em_distributions_1d}
    \end{figure}

    \vspace{-30pt}

    \begin{itemize}
      \item Derive $\emd$ from mean intensities in black boxes in Fig. \autoref{fig:aia_eis_intensities} using method of \citet{hannah_differential_2012} constrained by contribution functions from lines in Table 2 of \citet{warren_systematic_2012}.
      \item For $T\le T_\mathrm{peak}$, $\mathrm{EM}\sim T^a$---$a$ is the \alert{EM slope} \citep{jordan_structure_1975}
      \item Expect $2\lesssim a\lesssim2.5$ for uninterrupted radiative cooling \citep{cargill_implications_1994}; \alert{Larger $a$---HF heating; smaller $a$---LF heating}
      \item Fit $T^a$ between \SI{0.7}{\mega\kelvin} and $T_{peak}$--indicated by thick, solid line.
      \item \alert{No significant change in slope from one rotation to the next for all ARs.}
      \item \alert{ARs 12488 and 12674 show more hot plasma} in \textcolor{C0}{first} rotation compared to \textcolor{C1}{second} rotation.
    \end{itemize}

  \end{block}

\end{column}

\separatorcolumn

\begin{column}{\colwidth}

  \begin{block}{Emission Measure Slope}

    \begin{figure}
      \centering
      \begin{columns}[c]
        \begin{column}{0.57\colwidth}
          \import{figures/}{em_slope_maps.pgf}
        \end{column}
        \begin{column}{0.43\colwidth}
          \caption{Emission measure slopes computed in each EIS raster from the \textcolor{C0}{first}, \textcolor{C1}{second}, and \textcolor{C2}{third} rotations of NOAA 12674. $a$ is computed by first deriving the $\emd$ in each pixel of the raster, assuming an uncertainty of \SI{20}{\percent} on each line used to constrain the inversion. Then, $T^a$ is fit to the $\emd$ in each pixel between \SI{0.7}{\mega\kelvin} and $T_{peak}=\argmax_T{\emd}$. Pixels are masked and colored white if they were masked in the original raster in any line, fall below \SI{1e25}{\raiseto{-5}\centi\meter} anywhere in the fitting range, or if quality of the power-law fit does not meet the criteria $r^2<0.5$. This process is repeated for each AR shown in Fig. \autoref{fig:aia_eis_intensities}.}
          \label{fig:em_slope_maps}
        \end{column}
      \end{columns}
    \end{figure}

    \vspace{-35pt}

    \begin{figure}
      \centering
      \import{figures/}{em_slope_histograms.pgf}
      \vspace{-20pt}
      \caption{Histograms of $\emd$ slopes, $a$, from all unmasked pixels in each rotation of each AR studied. In each panel, the histograms are normalized as the EIS rasters may be of different sizes. Note that there \alert{is no consistent pattern in the evolution of the distribution of $a$ across these four ARs}.}
      \label{fig:em_slope_histograms}
    \end{figure}

  \end{block}

  \vspace{-35pt}

  \begin{block}{Time Lag Analysis}

    \vspace{-20pt}

    \begin{figure}
      \centering
      \begin{columns}[T]
        \begin{column}{0.65\colwidth}
          \import{figures/}{timelag_maps.pgf}
        \end{column}
        \begin{column}{0.35\colwidth}
          \caption{Maps of time lags for the 211-193 AIA channel pair for all three rotations of 12674 computed by finding time delay which maximizes cross-correlation between \SI{12}{\hour} intensity time series of \SIlist{211;193}{\angstrom} channels \citep{viall_evidence_2012}. Channels are ordered ``hot-cold'' such that a positive time lag implies cooling. Pixels with maximum cross correlation $<0.2$ are masked.}
          \label{fig:timelag_maps}
        \end{column}
      \end{columns}
    \end{figure}

    \vspace{-35pt}

    \begin{figure}
      \centering
      \import{figures/}{timelag_histograms.pgf}
      \caption{Distributions of unmasked time lags for three selected channel pairs (rows) for all four ARs (columns) over each rotation. The bin size is \SI{5}{\minute}. There is \alert{no clear trend in time lag evolution} in any channel pair over all four ARs, though \alert{12488 and 12682 show similar distributions in all channel pairs}.} 
      \label{fig:timelag_histograms}
    \end{figure}

  \end{block}

  \vspace{-35pt}

  \begin{block}{Potential Field Extrapolations}

    \vspace{-35pt}

    \begin{columns}[T]
      \begin{column}{0.333\colwidth}
        \begin{figure}
          \centering
          \import{figures/}{field_extrapolations_12674.pgf}
          \caption{Fieldlines traced from potential field extrapolation calculated from HMI LOS magnetogram for \textcolor{C0}{first} rotation of 12674 overlaid on an AIA \SI{171}{\angstrom} image.} 
          \label{fig:field_extrapolation}
        \end{figure}
      \end{column}
      \begin{column}{0.667\colwidth}
        \begin{figure}
          \centering
          \import{figures/}{fieldline_footpoint_ratios.pgf}
          \caption{Distributions of ratios of field strengths at opposite footpoints as calculated from a potential field extrapolation from each AR (e.g. Fig. \autoref{fig:field_extrapolation}). Loops with both footpoints rooted in strong field may exhibit ``quenched'' heating \citep{tiwari_new_2017}.} 
          \label{fig:fieldline_footpoint_ratios}
        \end{figure}
      \end{column}
    \end{columns}

  \end{block}

  \vspace{-25pt}

  \begin{block}{References}
    \scriptsize
    This research is funded by NASA grant 80NSSC24K0256 through the ROSES 2023 Heliophysics Supporting Research program.
    This research used v7.0.0 of \texttt{sunpy} \citep{the_sunpy_community_sunpy_2020}.
    \begin{multicols}{2}
      \bibliographystyle{aasjournal.bst}
      \bibliography{references.bib}
    \end{multicols}
  \end{block}

\end{column}

\separatorcolumn
\end{columns}
\end{frame}

\end{document}
