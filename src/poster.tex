% Gemini theme
% https://github.com/anishathalye/gemini

\documentclass[final]{beamer}

% ====================
% Packages
% ====================

\usepackage[T1]{fontenc}
\usepackage{lmodern}
% Lengths are in centimeters
\usepackage[size=a1, orientation=portrait, scale=1.0]{beamerposter}
\usetheme{gemini}
\usecolortheme{wtbarnes}
\usepackage{graphicx}
\usepackage{booktabs}
\usepackage{tikz}
\usepackage{pgfplots}
\pgfplotsset{compat=1.14}
\usepackage{anyfontsize}
\usepackage{multicol}
\usepackage[numbers]{natbib}
\usepackage{import}
\usepackage{siunitx}
\usepackage{float}

% ====================
% Custom Commands
% ====================
% This is to work around a bug in matplotlib that leaves a command undefined when
% using pgf figures: https://github.com/matplotlib/matplotlib/issues/27907
\def\mathdefault#1{\displaystyle #1}

% ====================
% Lengths
% ====================

% If you have N columns, choose \sepwidth and \colwidth such that
% (N+1)*\sepwidth + N*\colwidth = \paperwidth
\newlength{\sepwidth}
\newlength{\colwidth}
\setlength{\sepwidth}{0.0167\paperwidth}
\setlength{\colwidth}{0.45\paperwidth}
\newcommand{\separatorcolumn}{\begin{column}{\sepwidth}\end{column}}

% ====================
% Title
% ====================
\title{Diagnostics of Coronal Heating in Active Regions over Multiple Solar Rotations}
\author{
  W. T. Barnes \inst{1}\textsuperscript{,}\inst{2} \and
  S. J. Bradshaw \inst{3} \and
  E. M. Mason \inst{4} \and
  N. M. Viall \inst{2}
}
\institute[]{
  \inst{1} Department of Physics, American University \samelineand
  \inst{2} Heliophysics Science Division, NASA Goddard Space Flight Center \and
  \inst{3} Department of Physics and Astronomy, Rice University
  \inst{4} Predictive Science, Inc.
}

% ====================
% Footer (optional)
% ====================
\footercontent{
  \href{https://github.com/wtbarnes/hinode-2025-poster}{github.com/wtbarnes/hinode-2025-poster} \hfill
  Hinode 18/IRIS 16 Meeting --- London, UK --- 23--27 June 2025 \hfill
  \href{mailto:wbarnes@american.edu}{wbarnes@american.edu}
}

% ====================
% Logo (optional)
% ====================
\logoright{\includegraphics[height=7cm]{static/AU_stacked_logo.png}}
\logoleft{\includegraphics[height=8cm]{static/sunpy_logo_portrait_powered.png}}

% ====================
% Body
% ====================

\begin{document}

\begin{frame}[t]
\begin{columns}[t]
\separatorcolumn

\begin{column}{\colwidth}

  \begin{block}{Introduction}

    \begin{itemize}
      \item Heating frequency: \alert{frequency at which elemental strands in the corona are reenergized}
      \begin{itemize}
        \item High frequency (HF): waiting time between successive events less than a loop cooling time
        \item Low frequency (LF): waiting time between successive events greater than a loop cooling time
      \end{itemize}
      \item \alert{Heating frequency is variable over an active region (AR)} \citep[e.g.][]{del_zanna_evolution_2015,barnes_understanding_2021}
      \begin{itemize}
        \item Warm, periphery loops exhibit low-frequency heating \citep[e.g.][]{warren_evolving_2003}
        \item Hot, core loops more consistent with high-frequency heating \citep[e.g.][]{warren_evidence_2010}
      \end{itemize}
      \item \alert{AR properties evolve with age:} ``very hot'' emission \citep{ugarte-urra_is_2012,ugarte-urra_determining_2014}, abundance \citep{testa_coronal_2023}, $T$ and $n$ \citep{ko_correlation_2016}
      \item \alert{\textbf{Goal:}} Understand how the distribution of heating frequencies in an AR evolves with age
    \end{itemize}

  \end{block}

  \begin{block}{Observing Active Regions over Multiple Rotations}

    Give some details about how the regions were selected, what data were used

    Plot AIA images (171) at disk center along with 2 EIS rasters (hot and cool line) for all four regions for all rotations

    Show full-disk AIA images with EIS and AR bounding boxes?

  \end{block}

  \begin{block}{Differential Emission Measure Analysis}

    List what lines were used in the analysis, what method

    Show slope maps for 1-2 regions

    Show histograms of slopes for all 4 regions

    Maybe show DEM curves from selected subregion from each rotation of all regions? With slopes?

  \end{block}

\end{column}

\separatorcolumn

\begin{column}{\colwidth}

  \begin{block}{Time Lag Analysis}

    Very brief description of method, data used

    For a few selected channel pairs, show time lag maps for all regions

    For a few selected channel pairs, show histograms for all regions for all rotations overlaid

  \end{block}

  \begin{block}{Very Hot Plasma}
    
    Show isolated Fe XVIII emission from AIA

    Also show hottest EIS line for each region

  \end{block}

  \begin{block}{Magnetic Field Extrapolations}

    For one region, show evolution of potential field over rotations

    Show histograms of field strength over all traced field lines

    Show histograms of ratio of field strength between each footpoint

  \end{block}

  \begin{block}{Conclusions}

    Conclusions go here

  \end{block}

  \begin{block}{References}
    \scriptsize
    WTB was supported by NASA.
    This research used version 5.0.1 of the \texttt{sunpy} open source software package \citep{the_sunpy_community_sunpy_2020}.
    This research used verion 0.7 of the \texttt{aiapy} open source software package \citep{barnes_aiapy_2020}.
    \begin{multicols}{2}
      \bibliographystyle{aasjournal.bst}
      \bibliography{references.bib}
    \end{multicols}
  \end{block}

\end{column}

\separatorcolumn
\end{columns}
\end{frame}

\end{document}
